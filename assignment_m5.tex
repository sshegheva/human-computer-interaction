\documentclass[12pt,letterpaper]{article}

% just for the example
\usepackage{lipsum}
% Set margins to 1.5in
\usepackage[margin=1.5in]{geometry}
\usepackage[toc,page]{appendix}

% for graphics
\usepackage{graphicx}
\graphicspath{{./figures/m3/}}

% for crimson text
\usepackage{crimson}
\usepackage[T1]{fontenc}
\usepackage{url}

% setup parameter indentation
\setlength{\parindent}{0pt}
\setlength{\parskip}{6pt}

% for 1.15 spacing between text
\renewcommand{\baselinestretch}{1.15}

% For defining spacing between headers
\usepackage{titlesec}
% Level 1
\titleformat{\section}
  {\normalfont\fontsize{18}{0}\bfseries}{\thesection}{1em}{}
% Level 2
\titleformat{\subsection}
  {\normalfont\fontsize{14}{0}\bfseries}{\thesection}{1em}{}
% Level 3
\titleformat{\subsubsection}
  {\normalfont\fontsize{12}{0}\bfseries}{\thesection}{1em}{}
% Level 4
\titleformat{\paragraph}
  {\normalfont\fontsize{12}{0}\bfseries\itshape}{\theparagraph}{1em}{}
% Level 5
\titleformat{\subparagraph}
  {\normalfont\fontsize{12}{0}\itshape}{\theparagraph}{1em}{}
% Level 6
\makeatletter
\newcounter{subsubparagraph}[subparagraph]
\renewcommand\thesubsubparagraph{%
  \thesubparagraph.\@arabic\c@subsubparagraph}
\newcommand\subsubparagraph{%
  \@startsection{subsubparagraph}    % counter
    {6}                              % level
    {\parindent}                     % indent
    {12pt} % beforeskip
    {6pt}                           % afterskip
    {\normalfont\fontsize{12}{0}}}
\newcommand\l@subsubparagraph{\@dottedtocline{6}{10em}{5em}}
\newcommand{\subsubparagraphmark}[1]{}
\makeatother
\titlespacing*{\section}{0pt}{12pt}{6pt}
\titlespacing*{\subsection}{0pt}{12pt}{6pt}
\titlespacing*{\subsubsection}{0pt}{12pt}{6pt}
\titlespacing*{\paragraph}{0pt}{12pt}{6pt}
\titlespacing*{\subparagraph}{0pt}{12pt}{6pt}
\titlespacing*{\subsubparagraph}{0pt}{12pt}{6pt}

% Set caption to correct size and location
\usepackage[tableposition=top, figureposition=bottom, font=footnotesize, labelfont=bf]{caption}

% set page number location
\usepackage{fancyhdr}
\fancyhf{} % clear all header and footers
\renewcommand{\headrulewidth}{0pt} % remove the header rule
\rhead{\thepage}
\pagestyle{fancy}

% Overwrite Title
\makeatletter
\renewcommand{\maketitle}{\bgroup
   \begin{center}
   \textbf{{\fontsize{18pt}{20}\selectfont \@title}}\\
   \vspace{10pt}
   {\fontsize{12pt}{0}\selectfont \@author} 
   \end{center}
}
\makeatother

% Used for Tables and Figures
\usepackage{float}

% For using lists
\usepackage{enumitem}

% For using APA Citation format
\usepackage{apacite}

% Custom Quote
\newenvironment{myquote}[1]%
  {\list{}{\leftmargin=#1\rightmargin=#1}\item[]}%
  {\endlist}
  
% Create Abstract 
\renewenvironment{abstract}
{\vspace*{-.5in}\fontsize{12pt}{12}\begin{myquote}{.5in}
\noindent \par{\bfseries \abstractname.}}
{\medskip\noindent
\end{myquote}
}

\begin{document}

% Set Title, Author, and email
\title{Assignment M5}
\author{Snejana Shegheva \\ sshegheva3@gatech.edu}

\maketitle
\thispagestyle{fancy}

\begin{abstract}
Mapping data from one form to another for its ease-of-use is at the core of the \textit{Extract, Transform and Load} process. In this project, we evaluate various prototypes for an internal interface of a \textit{transform} task that prepares the data for use in a personalized recommendation system powered by Artificial Intelligence engines. The task involves altering the form of the received data with a purpose to extract relevant features in the form that is more advantageous for subsequent tasks. Our main goal is to assess all the weak areas of the suggested alternative models via early feedback collection from users who perform the task frequently and therefore have a good understanding of the task at hand.  
\end{abstract}

\section*{Qualitative Evaluation}
\subsubsection*{Think-aloud for Verbal Prototype}
\textit{Appendix} A contain verbal prototype that discusses a possible interface where a user is given a recommendation for which transformation function(s) is most compatible with the observed variable. The interface target both, novices who need to explore the system, and experts who aim to perform their tasks more efficiently.

\textbf{Evaluation Results}

The evaluation span across two sessions in the same day with the same participant. The first part took place in a casual setting of a short walk where I, as a "designer", engaged with the Product Manager in a conversation about the alternative design for the data transformation task. At the beginning of our conversation, I have set up the expectations for what I would like to get out of the discussion. The chosen participant has significant experience with the task at hand that contributed to a \textbf{productive session} given that the prototype was delivered verbally.

The second part of the evaluation happened a few hours later in the office setting. We choose to follow up on the remaining set of questions from the discussions that took place earlier. The participant expressed their feedback by sketching their thoughts on the piece of paper and elaborating on several analogies drawn from different tools and fields. Overall, the evaluation exceeded my expectations as it impacted how I was thinking about the design originally. 

\textbf{Think-Aloud Study}
To get started with the evaluation I described the idea to the user so they can imagine their workflow within the new interface. After they confirmed the understanding of new direction, I asked them to perform a specific transformation task while thinking aloud through the steps. \textit{Appendix B} contains the set of questions asked in the sessions.


\textbf{Feedback from the Participant}

Although the user liked the idea of a \textit{recommended} transformations, they were not convinced that this would increase the efficiency of the task. The main take-away was the recommendation should be focused on the \textit{outcome} instead of the \textit{function}. For example, if we have an unstructured field that contains \textit{dates}, recommending a transformation such as \textit{DateFinder} would be useful, however it does not communicate the result. If, instead, the interface suggests a few \textit{patterns} for what can be found 






\section*{Predictive Evaluation}

We are going to perform a \textbf{Cognitive Walkthrough} on the prototype shown in \textit{Appendix D}. The wireframes guide the user through the transformation task by keeping the sample of the data visible at all times. 

For the Cognitive Walkthrough, we need to identify the sequence of actions along with the series of questions that helps us evaluate the prototype.

\begin{itemize}
    \item \textbf{Step 1}: Select the variable to transform
    \begin{itemize}
        \item Does the user understand what does it mean for a variable to be selected?
        \item Can the user see how the selection is accomplished? Is it visible on the screen or it requires filtering and contextual menus?
        \item Is the source variable clearly labeled?
        \item If the user is unable to select a variable, and the feedback is given to the user, can they understand and apply the feedback? 
    \end{itemize}
    \item \textbf{Step 2}: Apply the desired transformation function
    \begin{itemize}
        \item Does the user have a correct conceptual model for that it means to change/transform the variable?
        \item Can the user see the list of available transformation functions?
        \item Do the functions' names reflect the clear operation? 
        \item If the user selects an incompatible transformation, is the feedback given to the user helpful and sufficient to correct their actions?
    \end{itemize}
    \item \textbf{Step 3}: Confirm the result for the transformation
    \begin{itemize}
        \item Does the user grasp how to evaluate the effect of the applied transformation?
        \item Can the user see what type of transformation function has been applied to the variable?
        \item Can the user infer if the transformation has been propagated (applied) or just saved (configured) on the screen?
        \item Can the user infer based on the feedback if the transformation is successful or not?
    \end{itemize}
\end{itemize}

The users will be accomplishing the sequence of steps described above in the context of the new interface. Although they would be experienced in the task of transformation, they would have to navigate in an unfamiliar interface and discover the methods required to complete the tasks. The questions defined in each sub-task can be used to evaluate the prototype and compare its usability between users with varying degree of experience on the task. 

The success of the prototype will be measured by the number of affirmative answers to the questions. For any item that has been answered as "NO," we will supply an explanation and a recommendation to resolve the gap. For example, if the user cannot find the functionality they look for because it is incorrectly labeled or not visible on the screen, the outcome of this evaluation will augment the prototype with additional features to complete the requirements. 

A limitation of the predictive evaluation is that for some of the steps we cannot get the affirmative answers because the interface does not exist yet. We cannot predict if the user would understand the feedback because then we do not show the actual feedback in the prototype. We might provide examples of possible feedback, and how and where it would be revealed, but we will not be able to anticipate the user's response. 

\section*{Preparing to Execute}

Without an actual interface, it will not be possible for us to carry out the Empirical Evaluation, mainly because it collects the quantitative data such as time elapsed between the start and the end of the task, and the number of steps taken to complete the task. Therefore, we postpone this evaluation plan for future iterations. 

The Qualitative Evaluation is typically performed in the early stages of prototyping. The feedback from users can inform the subsequent iterations of needfinding and prototyping. At this stage, it is essential to evaluate if the users' needs have been sufficiently met before proceeding to implementations that can be more costly regarding time and allocated resources. 

Since the Prediction Evaluation does not involve users, it is very \textit{cheap} and relatively efficient for informing ongoing design decisions, such as the positions and labeling of functionality required for accomplishing the task. This evaluation plan helps us formalize the user's thought process through the analysis of well-defined methods and operators. 

In conclusion, we choose to \textit{advance} the Qualitative and Predictive Evaluations, and \textit{drop} the Empirical Evaluation.  

\bibliographystyle{apacite} 
\bibliography{bibtemp}

\newpage
\section*{Appendices}

\appendix


\subsection*{A}

\subsection*{Prototype 1 - Verbal}

\textit{Imagine that you have an interface through which you can interact with the incoming streams of data. Your goal is to apply some rules to the received data dynamically. Those rules would transform the current and future data into the desired alternative output. You can treat these transformation rules as simple mathematical functions.}

\textit{The way the system currently works is by iterating through all variables, one at a time, giving the user an option to edit variable. To find a rule, you need to parse through a pile of currently available transformations to find the one most relevant to you. This, of course, could be a daunting task if transformations are not organized in any meaningful way.}

\textit{Imagine an alternative way of interacting with a system like that. Let's take as an example a case where you might be looking at a variable that contains a large chunk of text (maybe, transcripts from video lectures). The interface could recommend you a list of most likely transformations, such as:
\begin{itemize}
    \item Remove all punctuation (which could be useful for some subsequent NLP tasks)
    \item Extract keywords (instead of passing the entire text through the pipeline, you might be only interested in the topmost representative terms, and entities) 
\end{itemize}
}

\textit{The set you are given back is limited to the selected data type. For example, it won't recommend you transformations intended for numerical fields, such as Averaging, Adding/Subtracting, Computing rates, etc. This way your task of transforming can be optimized to your data leading to a more efficient workflow. What do you think?}

\subsection*{B}

\textbf{Think-Aloud Study}
To get started with the evaluation I described the idea to the user so they can imagine their workflow within the new interface. After they confirmed the understanding of new direction, I asked them to perform a specific transformation task while thinking aloud through the steps.

\textit{Question 1 - Use case}: Let's say you have collected data on the recently published papers in the field in Machine Learning and Artificial Intelligence. You have a field that contains a list of \textit{keywords}, and let's assume you only want to get the first-mentioned keyword from the list. How would you accomplish this task in the new interface?  

\textit{Question 2 - Use case}: Let's say that in the same data set, you have a field that captures the \textit{published date}. If you would like to split this into a \textit{year} and \textit{month} variable, how would you go about it?

\textit{Question 3 - Use case}: Let's say you have two \textit{date} fields - \textit{submission date} and \textit{acceptance date}. Your goal is to create a variable that stores the elapsed time between the two dates. Do you think the new interface will sufficiently guide you through the process? What do you imagine the interface should do if you are stuck on the task, such as - the recommendations for the transformations do not capture your intentions very well?   

\textit{Question 4 - General Satisfiability}: What tasks in your opinion are more efficiently performed in the new interface when compared to the current version? How about less efficiently? Would you still prefer the current interface and why? 

\textit{Question 5 - Learnability}: Do you expect a significant amount of guidance from the interface, and which areas? For example, would you be able to find the variables you need to change quickly? How quickly do you expect to see a transformation that suits your needs? Do you plan to request a different set of recommendations? Would you like to quickly access \textit{all} available transformations? 

\subsection*{C}

\subsection*{Prototype 2 - Textual}

\textbf{The Idea}.
Users are presented with a list of Transformations that allows them to apply the same rule to a set of fields in a \textit{single} instruction (think one-to-many as opposed to one-to-one described in the previous prototype)

\textbf{How it Works}. In order to perform any transformations on the input data, a user A) navigates to a dedicated Tab that lists transformation organized in alphabetical order by default; B) The user selects a transformation and is subsequently presented with the set of all available fields compatible with the chosen transformation; C) User highlights the areas they wish to apply the transformation to, and hits Save. Then repeat the process if other transformations are needed.

\textbf{Example}. \textit{User decides to clean the unstructured data by removing all special characters and maybe stop words as well. Users find a transformation "Remove Punctuation," and observes that half of the available fields for transformations become disabled (greyed out) since they represented numerical data or date ranges. From the subset of variables, the user selects data fields - "summary," "keywords," and "content" - and applies the selected transformation. Then user finds another transformation "Remove STOP words," and repeats that for the fields "summary."}

\subsubsection*{Functionality}

\begin{itemize}
    \item The page with the transformation \textbf{displays the top 10 choices} and provides a pagination mechanism in addition to a customization for how many results are shown on the page.
    \item The transformations are \textbf{searchable} by the names and their descriptions. 
    \item The transformations are \textbf{sortable} by relevance, recency, popularity, date of creation and date of update.
    \item Data fields available for a select transformation are \textbf{dynamic}. Incompatible fields are greyed out leaving user with a more manageable set.
    \item User is given an option to \textit{apply}, \textit{save}, and \textit{export} the current configuration in JSON format.
\end{itemize}

\subsection*{D}
\subsection*{Prototype 3 - Wireframes}
The task described in this project involves knowledge of data intended for transformations. Providing the user with a view that shows a sample of the data aligns well with the \textit{perceptability} principle that keeps the user informed about what is going on through appropriate feedback \cite{nielsen1994usability}. Figure~\ref{fig::2} start with a view of the data (similar to a DataFrame concept in pandas \cite{mckinney2011pandas})

\begin{figure}[h]
\centering
\includegraphics[scale=.3]{figures/m3/wireframe-screen1.png}
\caption{Screen 1: User selects a variable and is given a choice to either edit the existing variable or create a new one based on the selected. Data is retrieved via https://arxiv.org/'s API}
\label{fig::2}
\end{figure}

The user workflow can be described by: 

\begin{itemize}
    \item Select a variable that intended for a transformation. The variable name is highlighted (see Figure~\ref{fig::2}, step 1)
    \item A context menu appears that gives user an option to either \textit{Edit} the existing variable, or \textit{Add New}  (see Figure~\ref{fig::2}, step 2)
    \item Upon selecting to Add a new Variable based on the highlighted field, the user is given a list of Transformations to choose from (See Figure~\ref{fig::3})
    
\begin{figure}[h]
\centering
\includegraphics[scale=.3]{figures/m3/wireframe-screen2.png}
\caption{Screen 2: User is given a list of transformations to choose from}
\label{fig::3}
\end{figure}

    \item After the user selected a transformation, a \underline{new block} is created with \underline{a link} to the original variable. A user is prompted to give the new variable a name (see Figure~\ref{fig::4}). The \textit{attached} transformation is kept in the block to remind the user on the selected transformation rule. 
    
    
\begin{figure}[h]
\centering
\includegraphics[scale=.3]{figures/m3/wireframe-screen3.png}
\caption{Screen 3: A new block appears that is connected to the selected variable with a cursor prompt to name a new variable}
\label{fig::4}
\end{figure}

    \item Figure~\ref{fig::5} shows a possible view of the outcome of multiple transformations. The information is presented on one screen that given user feedback on the actions taken so far.

\begin{figure}[h]
\centering
\includegraphics[scale=.3]{figures/m3/wireframe-screen5.png}
\caption{Screen 4: A hypothetical view of the Directed Acyclical Graph (DAG) that is a results of user creating multiple transformations}
\label{fig::5}
\end{figure}

\end{itemize}

\end{document}
