\documentclass[12pt,letterpaper]{article}

% just for the example
\usepackage{lipsum}
% Set margins to 1.5in
\usepackage[margin=1.5in]{geometry}

% for graphics
\usepackage{graphicx}
\graphicspath{{./figures/p2/}}

% for crimson text
\usepackage{crimson}
\usepackage[T1]{fontenc}

% setup parameter indentation
\setlength{\parindent}{0pt}
\setlength{\parskip}{6pt}

% for 1.15 spacing between text
\renewcommand{\baselinestretch}{1.15}

% For defining spacing between headers
\usepackage{titlesec}
% Level 1
\titleformat{\section}
  {\normalfont\fontsize{18}{0}\bfseries}{\thesection}{1em}{}
% Level 2
\titleformat{\subsection}
  {\normalfont\fontsize{14}{0}\bfseries}{\thesection}{1em}{}
% Level 3
\titleformat{\subsubsection}
  {\normalfont\fontsize{12}{0}\bfseries}{\thesection}{1em}{}
% Level 4
\titleformat{\paragraph}
  {\normalfont\fontsize{12}{0}\bfseries\itshape}{\theparagraph}{1em}{}
% Level 5
\titleformat{\subparagraph}
  {\normalfont\fontsize{12}{0}\itshape}{\theparagraph}{1em}{}
% Level 6
\makeatletter
\newcounter{subsubparagraph}[subparagraph]
\renewcommand\thesubsubparagraph{%
  \thesubparagraph.\@arabic\c@subsubparagraph}
\newcommand\subsubparagraph{%
  \@startsection{subsubparagraph}    % counter
    {6}                              % level
    {\parindent}                     % indent
    {12pt} % beforeskip
    {6pt}                           % afterskip
    {\normalfont\fontsize{12}{0}}}
\newcommand\l@subsubparagraph{\@dottedtocline{6}{10em}{5em}}
\newcommand{\subsubparagraphmark}[1]{}
\makeatother
\titlespacing*{\section}{0pt}{12pt}{6pt}
\titlespacing*{\subsection}{0pt}{12pt}{6pt}
\titlespacing*{\subsubsection}{0pt}{12pt}{6pt}
\titlespacing*{\paragraph}{0pt}{12pt}{6pt}
\titlespacing*{\subparagraph}{0pt}{12pt}{6pt}
\titlespacing*{\subsubparagraph}{0pt}{12pt}{6pt}

% Set caption to correct size and location
\usepackage[tableposition=top, figureposition=bottom, font=footnotesize, labelfont=bf]{caption}

% set page number location
\usepackage{fancyhdr}
\fancyhf{} % clear all header and footers
\renewcommand{\headrulewidth}{0pt} % remove the header rule
\rhead{\thepage}
\pagestyle{fancy}

% Overwrite Title
\makeatletter
\renewcommand{\maketitle}{\bgroup
   \begin{center}
   \textbf{{\fontsize{18pt}{20}\selectfont \@title}}\\
   \vspace{10pt}
   {\fontsize{12pt}{0}\selectfont \@author} 
   \end{center}
}
\makeatother

% Used for Tables and Figures
\usepackage{float}

% For using lists
\usepackage{enumitem}

% For using APA Citation format
\usepackage{apacite}

% Custom Quote
\newenvironment{myquote}[1]%
  {\list{}{\leftmargin=#1\rightmargin=#1}\item[]}%
  {\endlist}
  
% Create Abstract 
\renewenvironment{abstract}
{\vspace*{-.5in}\fontsize{12pt}{12}\begin{myquote}{.5in}
\noindent \par{\bfseries \abstractname.}}
{\medskip\noindent
\end{myquote}
}

\begin{document}

% Set Title, Author, and email
\title{Project - Principles\\Analysis of JupyterLab Interface}
\author{Snejana Shegheva \\ sshegheva3@gatech.edu}

\maketitle
\thispagestyle{fancy}

\begin{abstract}
JupyterLab is a web-based interface for Project Jupyter that provides an interactive and reproducible computing platform\footnote{https://github.com/jupyterlab/jupyterlab}. In this project, we evaluate the interaction with Documents and Kernels, specifically with the Notebook plugin that serves an observable list of cells containing code, markdown, or raw data.  
\end{abstract}

\subsection*{Heuristic Evaluation}

First, perform a heuristic evaluation using the principles from Unit 2 on the interface as it currently exists. Answer the questions: what works well? What makes it work well? What doesn’t work well? Why doesn’t it work well? Make sure to address all these: even the worst interfaces usually have some things that work well. If you can’t think of any good things to say about the interface, select a different one: redesigning an interface with no positive elements at all would be too easy!

In writing this evaluation, it is critical that you ground your critiques in terms of the principles you have learned in Unit 2, both conceptually and using the same vocabulary. Your critique will primarily be evaluated based on how well it grounds its praise and criticism in the principles covered in Unit 2, and how accurately it leverages these principles. We would expect any strong answer to use at least five principles covered in Unit 2, where a ‘Principle’ can be nearly any topic from the unit, including any of the design principles, ideas like expert blindspot and learning curves, and concepts like gulfs of execution and evaluation. However, you are not limited to five principles, nor is five principles automatically sufficient if the individual principles are not leveraged with sufficient depth.

\subsection*{Interface Redesign}
Second, based on your evaluation, redesign the interface. If it’s a visual interface (such as a mobile app, visual wearable interface, or traditional desktop application), supply visual mock-ups of the potential redesign. If it’s a physical interface, supply a sketch or similar representation of the altered interface. If it’s an interface that cannot be provided in a 2D visual form like a haptic, auditory, or virtual reality interface, describe the redesign in words using diagrams wherever possible. Your redesign can contain textual annotations or rely on text if it is non-visual, but the text should merely explain the redesigned interface, not justify it.

\subsection*{Interface Justification}
Third, justify the redesigned interface. Describe how your redesigned interface addresses the criticisms from the first section, while preserving the positive elements of the original interface. Again, make sure to put your justification in terms of the principles covered in Unit 2, both conceptually and using the same vocabulary. You need not focus on the same principles covered in the previous section; you may, for example, leverage a particular principle to improve the interface even if it wasn’t explicitly violating that principle in the first place.

\bibliographystyle{apacite} 
\bibliography{bibtemp}

\end{document}
