\documentclass[12pt,letterpaper]{article}

% just for the example
\usepackage{lipsum}
\usepackage{url}
% Set margins to 1.5in
\usepackage[margin=1.5in]{geometry}

% for graphics
\usepackage{graphicx}
\graphicspath{{./figures/project-principles/}}

% for crimson text
\usepackage{crimson}
\usepackage[T1]{fontenc}

% setup parameter indentation
\setlength{\parindent}{0pt}
\setlength{\parskip}{6pt}

% for 1.15 spacing between text
\renewcommand{\baselinestretch}{1.15}

% For defining spacing between headers
\usepackage{titlesec}
% Level 1
\titleformat{\section}
  {\normalfont\fontsize{18}{0}\bfseries}{\thesection}{1em}{}
% Level 2
\titleformat{\subsection}
  {\normalfont\fontsize{14}{0}\bfseries}{\thesection}{1em}{}
% Level 3
\titleformat{\subsubsection}
  {\normalfont\fontsize{12}{0}\bfseries}{\thesection}{1em}{}
% Level 4
\titleformat{\paragraph}
  {\normalfont\fontsize{12}{0}\bfseries\itshape}{\theparagraph}{1em}{}
% Level 5
\titleformat{\subparagraph}
  {\normalfont\fontsize{12}{0}\itshape}{\theparagraph}{1em}{}
% Level 6
\makeatletter
\newcounter{subsubparagraph}[subparagraph]
\renewcommand\thesubsubparagraph{%
  \thesubparagraph.\@arabic\c@subsubparagraph}
\newcommand\subsubparagraph{%
  \@startsection{subsubparagraph}    % counter
    {6}                              % level
    {\parindent}                     % indent
    {12pt} % beforeskip
    {6pt}                           % afterskip
    {\normalfont\fontsize{12}{0}}}
\newcommand\l@subsubparagraph{\@dottedtocline{6}{10em}{5em}}
\newcommand{\subsubparagraphmark}[1]{}
\makeatother
\titlespacing*{\section}{0pt}{12pt}{6pt}
\titlespacing*{\subsection}{0pt}{12pt}{6pt}
\titlespacing*{\subsubsection}{0pt}{12pt}{6pt}
\titlespacing*{\paragraph}{0pt}{12pt}{6pt}
\titlespacing*{\subparagraph}{0pt}{12pt}{6pt}
\titlespacing*{\subsubparagraph}{0pt}{12pt}{6pt}

% Set caption to correct size and location
\usepackage[tableposition=top, figureposition=bottom, font=footnotesize, labelfont=bf]{caption}

% set page number location
\usepackage{fancyhdr}
\fancyhf{} % clear all header and footers
\renewcommand{\headrulewidth}{0pt} % remove the header rule
\rhead{\thepage}
\pagestyle{fancy}

% Overwrite Title
\makeatletter
\renewcommand{\maketitle}{\bgroup
   \begin{center}
   \textbf{{\fontsize{18pt}{20}\selectfont \@title}}\\
   \vspace{10pt}
   {\fontsize{12pt}{0}\selectfont \@author} 
   \end{center}
}
\makeatother

% Used for Tables and Figures
\usepackage{float}

% For using lists
\usepackage{enumitem}

% For using APA Citation format
\usepackage{apacite}

% Custom Quote
\newenvironment{myquote}[1]%
  {\list{}{\leftmargin=#1\rightmargin=#1}\item[]}%
  {\endlist}
  
% Create Abstract 
\renewenvironment{abstract}
{\vspace*{-.5in}\fontsize{12pt}{12}\begin{myquote}{.5in}
\noindent \par{\bfseries \abstractname.}}
{\medskip\noindent
\end{myquote}
}

\begin{document}

% Set Title, Author, and email
\title{Project - Principles\\Analysis of JupyterLab Interface}
\author{Snejana Shegheva \\ sshegheva3@gatech.edu}

\maketitle
\thispagestyle{fancy}

\begin{abstract}
JupyterLab is a web-based interface for Project Jupyter that provides an interactive and reproducible computing platform\footnote{https://github.com/jupyterlab/jupyterlab}. In this project, we evaluate the user interaction with Documents and Kernels, specifically with the Notebook plugin that serves an observable list of cells containing code, markdown, or raw data. Based on the evaluation results we suggest and justify interface improvements that can increase Data Scientists' productivity for creating reproducible computational narratives.
\end{abstract}

\subsection*{Heuristic Evaluation}
Figure~\ref{fig::1} shows an example of my current JupyterLab workspace, that include a file navigation section, interactive notebook area, additional consoles, and many other things that the platform provides. As a development environment, JupyterLab allows access to terminals, file viewers, notebooks, text editors - all from one place, which I personally find very convenient. With a rapid growth of data science and machine learning, Jupyter Notebooks have become a standard for creating reproducible computational narratives \cite{blog:jupyter}. 

\begin{figure}[h]
\centering
\includegraphics[scale=.5]{figures/project-principles/jupyter.png}
\caption{An example of JupyterLab Workspace.}
\label{fig::1}
\end{figure}

\subsubsection*{What works well}
Its most attractive feature - interactivity - thrives in both \textbf{gulfs}: \textbf{execution} and \textbf{evaluation}. The execution gulf is easily bridged as the cells provide a clear indication for where text/code should be entered. In the early stage of planning where the user has to choose an appropriate action out of all possible actions, the interface positions the blinking cursor at the beginning of a cell prompting the user to enter a code inside it. This follows the \textit{consistency} principle for designing interfaces where user can easily transfer the learning between different systems.

Compared to other more traditional IDEs (Integrated Development Environment), the Jupyter Lab has also a much narrower gulf of evaluation. \textit{Running} a cell yields immediate feedback that allows user assessing how well their intentions have been met. The output is displayed \textit{asynchronously} as it is generated in the Kernel allowing user to perceive, interpret, and compare the state of the execution with their goal or a sub-goal.

Cells are viewed as physical building blocks, and as such can be re-arranged using a \textbf{direct manipulation} via drag-n-drop operation. User is engaged in the process of completing their task by 1) planning the action, for example, hovering over the cell that needs to be moved; 2) grabbing the cell and dragging it over the area that specifies the new location; 3) releasing the hold when the cell is aligned with the desired destination. For the described task, the interface becomes \textit{invisible} and intuitive largely due to the tight relationships between two gulfs. As the user is working on their task of cell re-arrangement, the system provides a "hint" in form of target line for where the new cell would land before the action is complete. The \textit{feedback} is easy to interpret, and it matches with what the user is expecting to happen. Therefore, the goal can be accomplished more accurately (cell is in the correct position) and more efficiently (speed of execution) without the user focusing on the interface.


\subsubsection*{What doesn't works well}
Drag-n-dropping cells works pretty well when the distance between the positions (current and the desired) is small. As the distance increases, and the user has to scroll through the page while simultaneously "dragging" a cell, the \textbf{ease and comfort} principles are infringed that reduce the overall efficiency in accomplishing the task. 

Current implementation of the interface provides signifiers located in the toolbar for adding and deleting cells (see Figure~\ref{fig::2}). There are two issues with the chosen position for those two opposing operations: 1) by placing them \textit{next to each other}, the interface invites user errors such as \textbf{slips}; 2) by situating them on the toolbar the \textbf{gulf of execution} is widened at the stage of \textit{performing} an action.

The user has a correct \textit{mental model} on how to add or delete a cell as the icons for signifiers are self-explanatory and consistent. However, a user intending to do one action, for example, adding a cell, may inadvertently execute the opposing one - deletion. This is an example of \textit{action-based} slip that occurred due to a rushed user behavior. 

A sub-optimal interaction with executing the action (add/delete) presents itself again when the notebook is long enough that the user needs to move the mouse between the current position and the toolbar. This increases the \textit{gulf of execution} and decreases the efficiency especially for new users who have not learned the shortcuts for these commands.  

\begin{figure}[h]
\centering
\includegraphics[scale=.4]{figures/project-principles/jupyter_add_delete.png}
\caption{Aspect of the interface that allow Slip errors}
\label{fig::2}
\end{figure}

Another feature that contributes to a wide execution gulf is \textit{merging} cells that is a commonly requested action. The aspect of the interface that is designated for this functionality violates the \textbf{discoverability} principle. Unlike other operations, merging is accessible via \textit{Edit} Menu -> Merge Selected Cells and thus not easily discoverable. The contextual menu available at the right-click of the does not include the pointer to the merging operation. The mismatch between menus fails the consistency check \textit{within} the application, and further increases the gulf of execution. 

\begin{figure}[h]
\centering
\includegraphics[scale=.55]{figures/project-principles/cell_merge_eval.png}
\caption{Inefficiency of merging cells}
\label{fig::3}
\end{figure}


\iffalse
2.1 Introduction to Principles
2.2 Feedback Cycles
2.3 Direct Manipulation and Invisible Interfaces
2.4 Human Abilities
2.5 Design Principles and Heuristics
2.6 Mental Models and Representations
2.7 Task Analysis
2.8 Distributed Cognition
2.9 Interfaces and Politics
2.10 Conclusion to Principles

First, perform a heuristic evaluation using the principles from Unit 2 on the interface as it currently exists. Answer the questions: what works well? What makes it work well? What doesn’t work well? Why doesn’t it work well? Make sure to address all these: even the worst interfaces usually have some things that work well. If you can’t think of any good things to say about the interface, select a different one: redesigning an interface with no positive elements at all would be too easy!

In writing this evaluation, it is critical that you ground your critiques in terms of the principles you have learned in Unit 2, both conceptually and using the same vocabulary. Your critique will primarily be evaluated based on how well it grounds its praise and criticism in the principles covered in Unit 2, and how accurately it leverages these principles. We would expect any strong answer to use at least five principles covered in Unit 2, where a ‘Principle’ can be nearly any topic from the unit, including any of the design principles, ideas like expert blindspot and learning curves, and concepts like gulfs of execution and evaluation. However, you are not limited to five principles, nor is five principles automatically sufficient if the individual principles are not leveraged with sufficient depth.
\fi

\subsection*{Interface Redesign}

\begin{figure}[h]
\centering
\includegraphics[scale=.6]{figures/project-principles/cell_merge.png}
\caption{Mock-up for merging cells}
\label{fig::4}
\end{figure}

\begin{figure}[h]
\centering
\includegraphics[scale=.6]{figures/project-principles/cell_create.png}
\caption{Mock-up for ordering/deleting cells}
\label{fig::5}
\end{figure}

Second, based on your evaluation, redesign the interface. If it’s a visual interface (such as a mobile app, visual wearable interface, or traditional desktop application), supply visual mock-ups of the potential redesign. If it’s a physical interface, supply a sketch or similar representation of the altered interface. If it’s an interface that cannot be provided in a 2D visual form like a haptic, auditory, or virtual reality interface, describe the redesign in words using diagrams wherever possible. Your redesign can contain textual annotations or rely on text if it is non-visual, but the text should merely explain the redesigned interface, not justify it.

\subsection*{Interface Justification}
Third, justify the redesigned interface. Describe how your redesigned interface addresses the criticisms from the first section, while preserving the positive elements of the original interface. Again, make sure to put your justification in terms of the principles covered in Unit 2, both conceptually and using the same vocabulary. You need not focus on the same principles covered in the previous section; you may, for example, leverage a particular principle to improve the interface even if it wasn’t explicitly violating that principle in the first place.

\bibliographystyle{apacite} 
\bibliography{bibtemp}

\end{document}
