\documentclass[12pt,letterpaper]{article}

% just for the example
\usepackage{lipsum}
% Set margins to 1.5in
\usepackage[margin=1.5in]{geometry}

% for graphics
\usepackage{graphicx}
\graphicspath{{./figures/p3/}}

% for crimson text
\usepackage{crimson}
\usepackage[T1]{fontenc}

% setup parameter indentation
\setlength{\parindent}{0pt}
\setlength{\parskip}{6pt}

% for 1.15 spacing between text
\renewcommand{\baselinestretch}{1.15}

% For defining spacing between headers
\usepackage{titlesec}
% Level 1
\titleformat{\section}
  {\normalfont\fontsize{18}{0}\bfseries}{\thesection}{1em}{}
% Level 2
\titleformat{\subsection}
  {\normalfont\fontsize{14}{0}\bfseries}{\thesection}{1em}{}
% Level 3
\titleformat{\subsubsection}
  {\normalfont\fontsize{12}{0}\bfseries}{\thesection}{1em}{}
% Level 4
\titleformat{\paragraph}
  {\normalfont\fontsize{12}{0}\bfseries\itshape}{\theparagraph}{1em}{}
% Level 5
\titleformat{\subparagraph}
  {\normalfont\fontsize{12}{0}\itshape}{\theparagraph}{1em}{}
% Level 6
\makeatletter
\newcounter{subsubparagraph}[subparagraph]
\renewcommand\thesubsubparagraph{%
  \thesubparagraph.\@arabic\c@subsubparagraph}
\newcommand\subsubparagraph{%
  \@startsection{subsubparagraph}    % counter
    {6}                              % level
    {\parindent}                     % indent
    {12pt} % beforeskip
    {6pt}                           % afterskip
    {\normalfont\fontsize{12}{0}}}
\newcommand\l@subsubparagraph{\@dottedtocline{6}{10em}{5em}}
\newcommand{\subsubparagraphmark}[1]{}
\makeatother
\titlespacing*{\section}{0pt}{12pt}{6pt}
\titlespacing*{\subsection}{0pt}{12pt}{6pt}
\titlespacing*{\subsubsection}{0pt}{12pt}{6pt}
\titlespacing*{\paragraph}{0pt}{12pt}{6pt}
\titlespacing*{\subparagraph}{0pt}{12pt}{6pt}
\titlespacing*{\subsubparagraph}{0pt}{12pt}{6pt}

% Set caption to correct size and location
\usepackage[tableposition=top, figureposition=bottom, font=footnotesize, labelfont=bf]{caption}

% set page number location
\usepackage{fancyhdr}
\fancyhf{} % clear all header and footers
\renewcommand{\headrulewidth}{0pt} % remove the header rule
\rhead{\thepage}
\pagestyle{fancy}

% Overwrite Title
\makeatletter
\renewcommand{\maketitle}{\bgroup
   \begin{center}
   \textbf{{\fontsize{18pt}{20}\selectfont \@title}}\\
   \vspace{10pt}
   {\fontsize{12pt}{0}\selectfont \@author} 
   \end{center}
}
\makeatother

% Used for Tables and Figures
\usepackage{float}

% For using lists
\usepackage{enumitem}

% For using APA Citation format
\usepackage{apacite}

% Custom Quote
\newenvironment{myquote}[1]%
  {\list{}{\leftmargin=#1\rightmargin=#1}\item[]}%
  {\endlist}
  
% Create Abstract 
\renewenvironment{abstract}
{\vspace*{-.5in}\fontsize{12pt}{12}\begin{myquote}{.5in}
\noindent \par{\bfseries \abstractname.}}
{\medskip\noindent
\end{myquote}
}

\begin{document}

% Set Title, Author, and email
\title{Assignment P3}
\author{Snejana Shegheva \\ sshegheva3@gatech.edu}

\maketitle
\thispagestyle{fancy}

\subsection*{Question 1 - Design Principles and Heuristics}

\subsubsection*{Invisibility - Mapping}

An \textit{invisible} interface is typically described as an interaction system that \textit{naturally} aligns with the user's mental models \cite{blog:fastcompany}. By \textit{mapping} the user's flow using a technique of one-to-one spatial correspondence between the layout and the controls, the designers can achieve interface simplicity \cite{norman2013design}. Having clear mapping bridges the \textit{gulf of execution}, specifically in the early stage of \textit{planning} when the user has to choose an appropriate action out of all possible actions. Simplifying this step helps the user specify and execute the correct actions to achieve their goal. Although the principle is simple, it is not always easy to implement in practice because there could be cultural differences in how the world is represented. Variability in physical representations can be captured by combining the Mapping principle with other principles that can further constrain the actions. 

\subsubsection*{Invisibility - Constraints}
Another aspect of an invisible interface is discoverability that can be attained with a successful implementation of \textit{constraints} which limit as the set of all possible actions. After a user has planned their actions, they still need to determine \textit{how} to execute them. This ties into the next stage of the gulf of evaluation - \textit{specify}. Even if the interface is entirely new to the user, they should still be able to determine their course of actions. Constraints that reduce a set of possible operations across physical, cultural, semantic and logical domains are a most effective way to bridge the execution gulf in guiding the user to perform the right actions.

\subsubsection*{Invisibility - Feedback}
The accuracy of the interface interpretation depends on how well the user has perceived what happened in the world. Communicating the result of the action through the use of constant and immediate feedback helps to bridge the \textit{gulf of evaluation} via \textit{perception} stage. Properly designing the feedback mechanism leads to interfaces that are easy to follow and interact with; therefore planning its implementation should be given careful thought. Very frequently interfaces are designed with feedback that informs a user that something has happened, but without a necessary detail of \textit{what} exactly has happened. These interfaces exemplify poor design that is usually a result for saving costs \cite{norman2013design}. Too much feedback, however, can be obtrusive, especially if it requires confirmation that increases the need of user's attention to the interface. This reduces the \textit{invisibility} effect, and signals an incorrect implementation of the feedback mechanism.

\subsubsection*{Participant View - Tolerance}
Tolerance principle advocates for \textit{flexibility} in the design, and reduction in the cost of mistakes \cite{wiki:principles}. Humans are frequently blamed for wrong actions, or a lack of corrective actions without taking context into account ( recall the recent example of the false alert of the ballistic missile threat \cite{blog:fastcompany_hawaii} - see Figure~\ref{fig::0}). Understanding \textit{why} an error occurs should lead not to punishment, but to a better design with an attention to tolerance, especially to high-cost errors. The context becomes very important, and the interface design should assume that an action can be taken under severe emotional stress and/or in a rush.  

\begin{figure}[h]
\centering
\includegraphics[scale=.35]{figures/p3/tolerance.png}
\caption{Interface of the Screen in the Hawaii False Alert Case. }
\label{fig::0}
\end{figure}

\subsubsection*{Participant View - Affordances}
Affordances represent possibilities of the interaction with the interface and can be perceived or invisible. Norman brings an example of public doors that open outward with a panic bar that serves as an affordance. This is a great example for emphasizing the user's context when designing an interface. Signifiers are frequently used together with affordances, for example labeling the door as \textit{push} or \textit{pull}, however, in extreme situations (such as a fire in the building), reading and perceiving the message may not be feasible, therefore exposing the user to dire consequences. Affordances should help specify \textit{what} to do and \textit{where} with as little as possible assumption on the rational state of the user who is interacting with the interface. 

\subsection*{Question 2 - Constraint, Mappings and Affordances Principles}

Although, not entirely an activity from my everyday life, I used to participate regularly in an adrenaline rushing practice of flying trapeze. For safety reasons, all students must wear a belt harness. When on the platform, safety lines are attached to the belt so that a trained professional can reduce the impact of wrong falling. Figure~\ref{fig::1} demonstrates attaching a carabiner (that holds safety lines) to a belt in two ways: \textit{overhook} vs \textit{underhook}\footnote{The terms \textit{overhook} vs. \textit{underhook} in this context have nothing to do with wrestling}. The preferred way is to use \textit{overhook} position that greatly simplifies the process of detachment. Although the \textit{underhook} position is not wrong, it makes it very inconvenient for your wrists to twist the lock to open and release. It is very \textbf{easy} to hook the carabiners from safety lines to the belt harness in a sub-optimal way because there is nothing to prevent you from hooking it in either direction (and both do their job of protecting you from a fall). The \textbf{penalty} here comes \textit{after} completing the jump where you have to release the safety lines from both your sides. It takes a lot of wiggling the carabiners around to put them back in the position where it is possible to unlock them quickly and safely for subsequent transfer back to the person on the board.   

\begin{figure}[h]
\centering
\includegraphics[scale=.35]{figures/p3/flying_trapeze.png}
\caption{Demonstration of two ways to attach a carabiner to a belt. Left: preferred. Right: Inconvenient (especially for detaching using one hand.)}
\label{fig::1}
\end{figure}

\subsubsection*{Constraints}
One clue for limiting the wrong action is to modify the ring on the belt that constrains you from passing the hook unless it is a specific position. For example, changing the thickness of the rings in different areas can guide the proper clasping position. Alternatively, the hook on the carabiner itself can have a more intricate shape that forces you to angle it in a certain way.

\subsubsection*{Mappings}
Don Norman defines a  \textit{mappings} design principle in terms of relationship a control and resulting function\cite{norman2013design}. The result of a carabiner locking depends on how the person originally holds it. Therefore, a mapping principle suggests adding visual clues for the desired alignment between the hand and the carabiner. Similarly to how some shipping packages are labeled with an arrow pointing up, the surface of the carabiner can be engraved with an arrow to hint on the direction of a hold. Adding a visual cue does not constrain the user from doing it wrong anyway, but it at least helps planning the action and increases the chances for getting it right.

\subsubsection*{Affordances}
A strong relative to mapping principle is \textit{affordance} principle that determines the interaction method between the agent and the interface \cite{norman2013design}. To improve the carabiner interface, the shape can be mapped to the shape of the palm. When we clasp our hands, it is more natural the grip to be wider closer to the thumb, and more narrow towards the pinkie. If the carabiner for flying trapeze can resemble a \textit{pear} shape, it would be more intuitive to grasp the carabiner in a certain way that it is more likely for you to hook it from a right angle. The shape, in this case, affords a proper handle \textit{before} the carabiner is attached, therefore determining the best action prior to execution.


\subsection*{Question 3 - Slips, Mistakes, and Errors}
Yousican is an innovate application with elements of game playing that allows students to learn how to play piano, guitar, and other musical instruments\cite{eli2017yousician}. It is a fun game that involves following step-by-step tutorials, exercises, and play-alongs. At the core of learning to play a musical instrument is a lot of practice, and slips and mistakes are an unavoidable part of the experience. Figure~\ref{fig::2} shows an example of a screen that user interacts with.

\begin{figure}[h]
\centering
\includegraphics[scale=.35]{figures/p3/yousician.png}
\caption{A screenshot from Yousician App that teachers the user to play piano}
\label{fig::2}
\end{figure}

\subsubsection*{Slips}
When a user follows along by playing the given music score, they may inadvertently hit a wrong note. The player might have had a correct intention, but the action led to a \textit{slip}. Don Norman classifies slips into two categories: action-based and memory-based \cite{norman2013design}. Both types are applicable here and describe two different reasons for playing a wrong note(s). In the action-based slip, the tempo of the music might be too fast for the user's dexterity level causing them to either miss a note or hit a note that is one or more semitones away from the original. In the memory-based slip, the user might have a memory lapse, that causes them to miss one or more notes even though they had learned the section correctly. While it can be argued that the interface should not change (since only practice can reduce slips), the app can learn the user's tendencies and frequencies of hitting wrong notes, and \textit{intelligently} adjust the tempo when needed that can address action-based slips. The best way to improve the memory-based slips is to help the user map and \textit{understand} the music passages. For example, highlight the notes that make up the chord, label the progression, link to the theory behind the music piece, etc.         

\subsubsection*{Mistake}
Playing wrong note can be classified as a \textit{mistake} if there is a pattern for erring in playing the passage. For example, if a signature of the music piece is labeled as G major (all F's are sharp - "\#"), and the user plays all the F's natural instead, this clearly points to a \textit{rule-based} mistake as the user has incorrectly evaluated the accidentals associated with the piece. An interface can be improved by pausing the auto-play and providing a visual animation that shows raising a finger by a semitone to indicate the correct rule for the given signature.

Both improvements in the interfaces leverage constraints (slow down, pause, stop) and mappings (overlay chord schemes, show finger position) to correct user's conceptual model in case of mistakes and user's tendencies to occasionally slip.

\subsubsection*{Challenge}
The game currently provides fourteen levels for piano playing ranging from elementary to advanced. A user selecting a level beyond their current abilities is likely to fail at adequately executing the music piece. These type of errors are not characterized as either slips or mistakes. Correcting the user's actions may not lead to any improvements if the user does not currently possess the necessary knowledge and experience (pre-requisites) to play through a challenging piece. Therefore, the only viable option here is to keep practicing. Practice makes perfect!

\subsection*{Question 4 - Models and Representations}

\subsubsection*{Duolingo}
Duolingo is a science-based app that lets the user learn a language for free by providing many topics, discussions, and recently podcasts \cite{von2013duolingo}. Figure~\ref{fig::3} shows a screenshot of the user's "home page" for the given language. A good representation of the mental model can help the user navigate the interface effectively to optimize their learning experience. Designers of the Duolingo App chose visual imagery for various topics (for example, an image of a phone for \textit{Work} subject, an image of a heart for \textit{Emotions} subject, etc.) This exemplifies the criteria of \textbf{predictability} that allows the user to predict the nature of the questions and vocabulary upon selecting a "circle." Another aspect of good representation for learnability is \textbf{synthesizeability} that allows the user to assess the effect of their past actions. Each circle denoting a subject has an outer ring that represents how much the user has completed for the given subject, and how much more to go.

\begin{figure}[h]
\centering
\includegraphics[scale=.6]{figures/p3/duolingvo.png}
\caption{A Screenshot from Duolingo App for learning a Language}
\label{fig::3}
\end{figure}

\subsubsection*{Automatic Soap Dispensers}
In the last year, I had traveled more frequently both internationally and locally, so I have been in multiple airports and observed a few variations for the installed automatic soap dispensers in the restrooms. An automatic dispenser is when a person places their hands under the nozzle that activates a sensor and dispenses a controlled amount of soap without having to pump it manually. There exist multiple types of sensors (radar, photo, infrared, etc.); however, this should be completely  \textit{invisible} to the user, and an action required for "requesting" soap should be consistent across device types. I expect external \textbf{consistency} between different brands of soap dispensers, so I don't have to wave my hands horizontally, vertically, sideways, and frequently on my neighbour's side. A violation for representation of mental models is a lack of \textbf{predictability} in the case when the dispenser is empty. There is no visible indication if the dispenser is malfunctioning or merely empty. 

\bibliographystyle{apacite} 
\bibliography{bibtemp}

\end{document}
