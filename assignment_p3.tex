\documentclass[12pt,letterpaper]{article}

% just for the example
\usepackage{lipsum}
% Set margins to 1.5in
\usepackage[margin=1.5in]{geometry}

% for graphics
\usepackage{graphicx}
\graphicspath{{./figures/p3/}}

% for crimson text
\usepackage{crimson}
\usepackage[T1]{fontenc}

% setup parameter indentation
\setlength{\parindent}{0pt}
\setlength{\parskip}{6pt}

% for 1.15 spacing between text
\renewcommand{\baselinestretch}{1.15}

% For defining spacing between headers
\usepackage{titlesec}
% Level 1
\titleformat{\section}
  {\normalfont\fontsize{18}{0}\bfseries}{\thesection}{1em}{}
% Level 2
\titleformat{\subsection}
  {\normalfont\fontsize{14}{0}\bfseries}{\thesection}{1em}{}
% Level 3
\titleformat{\subsubsection}
  {\normalfont\fontsize{12}{0}\bfseries}{\thesection}{1em}{}
% Level 4
\titleformat{\paragraph}
  {\normalfont\fontsize{12}{0}\bfseries\itshape}{\theparagraph}{1em}{}
% Level 5
\titleformat{\subparagraph}
  {\normalfont\fontsize{12}{0}\itshape}{\theparagraph}{1em}{}
% Level 6
\makeatletter
\newcounter{subsubparagraph}[subparagraph]
\renewcommand\thesubsubparagraph{%
  \thesubparagraph.\@arabic\c@subsubparagraph}
\newcommand\subsubparagraph{%
  \@startsection{subsubparagraph}    % counter
    {6}                              % level
    {\parindent}                     % indent
    {12pt} % beforeskip
    {6pt}                           % afterskip
    {\normalfont\fontsize{12}{0}}}
\newcommand\l@subsubparagraph{\@dottedtocline{6}{10em}{5em}}
\newcommand{\subsubparagraphmark}[1]{}
\makeatother
\titlespacing*{\section}{0pt}{12pt}{6pt}
\titlespacing*{\subsection}{0pt}{12pt}{6pt}
\titlespacing*{\subsubsection}{0pt}{12pt}{6pt}
\titlespacing*{\paragraph}{0pt}{12pt}{6pt}
\titlespacing*{\subparagraph}{0pt}{12pt}{6pt}
\titlespacing*{\subsubparagraph}{0pt}{12pt}{6pt}

% Set caption to correct size and location
\usepackage[tableposition=top, figureposition=bottom, font=footnotesize, labelfont=bf]{caption}

% set page number location
\usepackage{fancyhdr}
\fancyhf{} % clear all header and footers
\renewcommand{\headrulewidth}{0pt} % remove the header rule
\rhead{\thepage}
\pagestyle{fancy}

% Overwrite Title
\makeatletter
\renewcommand{\maketitle}{\bgroup
   \begin{center}
   \textbf{{\fontsize{18pt}{20}\selectfont \@title}}\\
   \vspace{10pt}
   {\fontsize{12pt}{0}\selectfont \@author} 
   \end{center}
}
\makeatother

% Used for Tables and Figures
\usepackage{float}

% For using lists
\usepackage{enumitem}

% For using APA Citation format
\usepackage{apacite}

% Custom Quote
\newenvironment{myquote}[1]%
  {\list{}{\leftmargin=#1\rightmargin=#1}\item[]}%
  {\endlist}
  
% Create Abstract 
\renewenvironment{abstract}
{\vspace*{-.5in}\fontsize{12pt}{12}\begin{myquote}{.5in}
\noindent \par{\bfseries \abstractname.}}
{\medskip\noindent
\end{myquote}
}

\begin{document}

% Set Title, Author, and email
\title{Assignment P3}
\author{Snejana Shegheva \\ sshegheva3@gatech.edu}

\maketitle
\thispagestyle{fancy}

\subsection*{Question 1 - Design Principles and Heuristics}

\subsection*{Question 2 - Constraint, Mappings and Affordances Principles}

Although, not entirely an activity of my everyday life, I used to regularly participate in an adrenaline-raising practice of flying trapeze. For safety reasons, all students must wear protection gear that includes a belt. When on the platform, safety lines are attached to the belt so that trained professionals can reduce the impact of wrong falling.    

\subsubsection*{Constraints}

\subsubsection*{Mappings}

\subsubsection*{Affordances}
Flying Trapeze

Constraint - do not let the carabine hook upside donwn 
Mapping - map the shape of the carabine to the hand 

\subsection*{Question 3 - Slips, Mistakes and Errors}

Yousician 

Slip: missing a note 
Mistake: Hitting an incorrect note (ex, missing sharps in G major)

First, describe a slip that a player of the game might make. Remember, a slip generally occurs when the player knows what action they should take, but does something different instead. In Tetris, this might be a player wanting to move a piece to the right, but pressing the left button instead. Then, describe why the player might make that slip. Then, briefly suggest a way the interface could be changed to prevent that slip in the future.

Second, describe a mistake that a player of the game might make. Remember, a mistake generally occurs when the player knows what they want to accomplish, but doesn’t know how to actually make it happen. In Tetris, this might be a player wanting to rotate a piece clockwise, but pressing to rotate it counter-clockwise instead because they do not know which button rotates clockwise. Then, describe why the player might make that mistake. Then, briefly suggest a way the interface could be changed to prevent that mistake in the future.

Finally, describe something that makes the game challenging, but that is not a slip or a mistake. For example, in Tetris, there may be no obvious place for a piece to go, but that does not force the user to commit a slip or a mistake.

\subsection*{Question 4 - Representation of the underlying content}

\bibliographystyle{apacite} 
\bibliography{bibtemp}

\end{document}
